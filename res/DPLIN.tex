\tdplotsetmaincoords{60}{110}

%define polar coordinates for some vector
%TODO: look into using 3d spherical coordinate system
\pgfmathsetmacro{\rvec}{.5}
\pgfmathsetmacro{\thetavec}{30}
\pgfmathsetmacro{\phivec}{60}

%start tikz picture, and use the tdplot_main_coords style to implement the display 
%coordinate transformation provided by 3dplot
\begin{tikzpicture}[scale=5,tdplot_main_coords]

	%set up some coordinates 
	%-----------------------
	\coordinate (O) at (0,0,0);

	%determine a coordinate (P) using (r,\theta,\phi) coordinates.  This command
	%also determines (Pxy), (Pxz), and (Pyz): the xy-, xz-, and yz-projections
	%of the point (P).
	%syntax: \tdplotsetcoord{Coordinate name without parentheses}{r}{\theta}{\phi}
	\tdplotsetcoord{P}{\rvec}{\thetavec}{\phivec}

	%draw figure contents
	%--------------------

	%draw the main coordinate system axes
	\draw[thick,->] (0,0,0) -- (.5,0,0) node[anchor=north east]{$x$};
	\draw[thick,->] (0,0,0) -- (0,.5,0) node[anchor=north west]{$y$};
	\draw[thick,->] (0,0,0) -- (0,0,.5) node[anchor=south,pos=0.95]{$z$};

	%draw a vector from origin to point (P) 
	\draw[-stealth,color=red,thick] (O) -- (P) node[midway, below]{$\vec{r} $};
	\draw[-stealth,color=green,thick] (P) -- +(0,0,0.2) node[above]{$ \vec{\ubar{A}}= \ubar{A}_z\vu{z}$};

	%draw projection on xy plane, and a connecting line
	\draw[dashed, color=red] (O) -- (Pxy);
	\draw[dashed, color=red] (P) -- (Pxy);

	%draw the angle \phi, and label it
	%syntax: \tdplotdrawarc[coordinate frame, draw options]{center point}{r}{angle}{label options}{label}
	\tdplotdrawarc{(O)}{0.2}{0}{\phivec}{anchor=north}{$\varphi$}

	%set the rotated coordinate system so the x'-y' plane lies within the
	%"theta plane" of the main coordinate system
	%syntax: \tdplotsetthetaplanecoords{\phi}
	\tdplotsetthetaplanecoords{\phivec}

	%draw theta arc and label, using rotated coordinate system
	\tdplotdrawarc[tdplot_rotated_coords]{(0,0,0)}{0.3}{0}{\thetavec}{anchor=south west}{$\vartheta$}

	%draw some dashed arcs, demonstrating direct arc drawing
	%\draw[dashed,tdplot_rotated_coords] (\rvec,0,0) arc (0:90:\rvec);
	%\draw[dashed] (\rvec,0,0) arc (0:90:\rvec);

	%set the rotated coordinate definition within display using a translation
	%coordinate and Euler angles in the "z(\alpha)y(\beta)z(\gamma)" euler rotation convention
	%syntax: \tdplotsetrotatedcoords{\alpha}{\beta}{\gamma}
	\tdplotsetrotatedcoords{\phivec}{\thetavec}{0}

	%translate the rotated coordinate system
	%syntax: \tdplotsetrotatedcoordsorigin{point}
	\tdplotsetrotatedcoordsorigin{(P)}

	%use the tdplot_rotated_coords style to work in the rotated, translated coordinate frame
	\draw[thin,tdplot_rotated_coords,->] (0,0,0) -- (.1,0,0) node[anchor=north west]{$\vu{\vartheta}$};
	\draw[thin,tdplot_rotated_coords,->] (0,0,0) -- (0,.1,0) node[anchor=west]{$\vu{\varphi}$};
	\draw[thin,tdplot_rotated_coords,->] (0,0,0) -- (0,0,.1) node[anchor=south]{$\vu{r}$};

	\node (a) [cylinder, shape border rotate=90, draw, minimum height=5mm, minimum width=2mm,yshift=-1mm] {};
	\draw [<->] ([xshift=-2pt]a.before bottom) -- ([xshift=-2pt]a.after top) node [midway, left] {$\ell$};

\end{tikzpicture}